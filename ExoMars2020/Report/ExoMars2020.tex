\documentclass[12pt,a4paper]{article}
\usepackage{fontspec}
\defaultfontfeatures{Mapping=tex-text}
\usepackage{xunicode}
\usepackage{xltxtra}
\setmainfont{FreeSerif}
\setmonofont{FreeMono}
\usepackage{polyglossia}
\setdefaultlanguage{english}
\usepackage{amsmath}
\usepackage{amsfonts}
\usepackage{amssymb}
\usepackage{siunitx}
\usepackage[margin=2.5cm]{geometry}

\author{Anas Naïri \and Julien Plante}
\title{Modelisation of ExoMars 2020's network using SystemC}


\begin{document}
\maketitle
\section{Task formulation}


\section{Devices description}
\subsection{NavCam}
\subsection{PanCam}
\subsection{ISEM}
\subsection{CLUPI}
\subsection{WISDOM}
\subsection{Adron}
\subsection{Ma\_MISS}
Ma\_MISS, for Mars Multispectral Imager for Subsurface Studies, is a spectrometer located in the drill of the rover, used to determine horizontal and vertical composition of the Martian soil. By illuminating the borehole and analysing the reflected light and its spectrum, it will be able to gather information about the distribution of minerals, especially of water-related ones, to search potential indicators of life.

It will work alongside the three other spectrometers (RLS, MicrOmega, MOMA), being specialised in studying unexposed material, and in collaboration with WISDOM and ADRON to choose interesting drilling location

\subsection{MicrOmega}

MicrOmega is an infra-red spectrometer made to identify composition of Martian soil samples at a grain scale, after their gathering by the drilling system. It is similar as RLS and MOMA in this way, since the three spectrometers will study samples collected by the drill. The infra-red study of the samples is adapted to find evidences of past or present carbon and water presence. 
It uses an infra-red hyper-spectral microscopic imager to acquire the spectrum of a 250$\times$256 pixels square (\char`\~5$\times$5 \si{\milli\metre^2}) for 320 wave length, between 0.95 and 3.65 \si{\micro\metre}. Thus having a maximum of \num{20480000} bytes to transmit.

\subsection{RLS}
RLS uses the Raman effect to find life signatures in Martian soil samples in a non-destructive way.
The measurements carried out by the RLS will be performed as described within the ExoMars Rover Reference Surface Mission, which includes six experiment cycles (with two samples each, one extracted from a surface target and the other at depth) and two vertical surveys (with five samples each extracted at different depths). It generates information about a 2048$\times$512 pixels of 15 \si{\micro\metre}, totalling a surface of approximately 30.7$\times$7.7 \si{\micro\metre^2}, and \num{1048576} bytes.

\subsection{MOMA}
MOMA (Mars Organics Molecule Analyser) is an instrument designed to detect organic molecules in spots of interest detected by the collaboration of RLS and MicrOmega. To do so, it will study samples gathered by the drill, as well as analysing the gases of the Martian atmosphere. It features to modes of operation : Gas Chromatograph-Mass Spectrometry (MOMA GC-MS) and Laser Desorption-Mass Spectrometry (MOMA LD-MS), the first one being used to analyse atmosphere gases, and the second soil samples. No information about size of data produced was found, 


\pagebreak
\nocite{*}
\bibliographystyle{unsrt}
\bibliography{references}
\end{document}